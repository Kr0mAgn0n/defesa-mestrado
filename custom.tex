\newcommand{\vectorfieldsspace}[1]{
	\mathfrak{X}(#1)
}

\newcommand{\normalvectorfieldsspace}[1]{
	\mathfrak{N}(#1)
}

\newcommand{\smoothfunctionsspace}[1]{
	C^\infty(#1)
}

\newcommand{\innerproduct}[2]{
	\left\langle #1,#2 \right\rangle
}

\newcommand{\realnumbers}{
	\mathbb{R}
}


\newcommand{\xb}{
	\overline{x}	
}



\newcommand{\yb}{
	\overline{y}	
}

\newcommand{\R}{\mathbb R}
\newcommand{\Sp}{\mathbb{S}}


\newcommand{\pdiff}[1]{
	\frac{\partial}{\partial #1}
}



\newcommand{\partialdiff}[2]{\frac{\partial #1}{\partial #2}}

\newcommand{\npartialdifffrac}[3]{\frac{\partial^#3 #1}{\partial #2^#3}}





\newcommand{\liebrackets}[2]{
	\left[ #1, #2 \right]
}



\newcommand{\curvaturetensor}[3]{
	\mathcal{R} \left( #1,#2 \right) #3
}

\newcommand{\norm}[1]{
	\left| #1 \right|
}


\newcommand{\complexnumbers}{
	\mathbb{C}
}

\newcommand{\N}{
	\mathbb{N}
}

\newcommand{\euclideanconnection}{
	\overline{\nabla}
}

\newcommand{\tangentialconnection} {
	\nabla^\top
}

\newcommand{\parentheses}[1]{
	\left( #1 \right)
}

\DeclareMathOperator\supp{supp}
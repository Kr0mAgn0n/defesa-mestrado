\documentclass[12pt,a4paper]{beamer}
\usepackage[utf8]{inputenc}
\usepackage[T1]{fontenc}
\usepackage{amsmath}
\usepackage{amsfonts}
\usepackage{amssymb}
\usepackage{graphicx}
\usepackage[portuguese]{babel}
\usepackage{amsthm}
\usepackage{amsmath}
\usepackage[alf]{abntex2cite}
\usepackage{parskip}

\author{Mario Alexis Lamas Espinoza}
\title{Conjectura de Lawson}
\setbeamertemplate{theorems}[numbered]

\newtheorem{teorema}{Teorema}
\newtheorem{proposicao}{Proposição}
\theoremstyle{definition}
\newtheorem{definicao}{Definição}
\newtheorem{observacao}{Observação}


\include{custom}

\begin{document}

\begin{frame}
	\maketitle	
\end{frame}

\section{Conjectura de Lawson}

\begin{frame}
	\frametitle{Toro de Clifford}
	
	\begin{definicao}
		Seja $f: \R^2 \rightarrow S^3$ definida por
		\begin{equation*}
			f(u,v) = \frac{1}{\sqrt{2}} \left(\cos u, \sin u, \cos v, \sin v\right).
		\end{equation*}
		$f$ é chamada de \emph{toro de Clifford}.
	\end{definicao}

	\begin{observacao}
		$f$ é um toro porque é congruente com $S^1 \left(\frac{1}{\sqrt{2}}\right) \times S^1 \left(\frac{1}{\sqrt{2}}\right)$.
	\end{observacao}
\end{frame}

\begin{frame}
	\frametitle{Conjectura de Lawson}
	\begin{teorema}
		Seja $F: \Sigma \rightarrow S^3$ um mergulho mínimo do toro em $S^3$. Então a imagem de $F$ é congruente ao toro de Clifford.
	\end{teorema}
\end{frame}





\begin{frame}
	\begin{proposicao}[\cite{Brendle2013}]
		\label{sup-min-nao-tem-pontos-umbilicos}
		Uma superfície mínima imersa em $S^3$ de genus 1 não tem pontos umbílicos, i.e., a segunda forma fundamental não é zero em tudo ponto da superfície.
	\end{proposicao}
\end{frame}


\begin{frame}

	Definamos $\Psi: \Sigma \rightarrow \R$ tal que
	\begin{equation*}
	\Psi(x) = \frac{\norm{A(x)}}{\sqrt{2}},
	\end{equation*}
	onde $A$ é a segunda forma fundamental.

	\begin{observacao}
		Pela Proposição \ref{sup-min-nao-tem-pontos-umbilicos}, $\Psi(x) \neq 0$.
	\end{observacao}

\end{frame}

\begin{frame}
	
	
	\begin{teorema}[\cite{Lawson1969}]
		\label{curv-gauss-de-sup-min-em-S3}
		Se $M^2$ é uma superfície mínima em $S^3$ de curvatura Gaussiana constante $K$, então $K=1$ e $M^2$ é totalmente geodésica, ou $K=0$ e $M^2$ é um pedaço aberto do toro de Clifford.
	\end{teorema}	
	
\end{frame}

\begin{frame}
	\begin{proposicao}\label{aleph-leq-1}
		Se
		\begin{equation*}
			\sup_{\substack{x,y \in \Sigma \\ x \neq y}} \frac{\norm{\innerproduct{\nu(x)}{F(y)}}}{\Psi(x) (1-\innerproduct{F(x)}{F(y)})} \leq 1,
		\end{equation*}
		então $F$ é congruente ao toro de Clifford.
	\end{proposicao}
\end{frame}

\begin{frame}
	\begin{itemize}
		\item Da desigualdade do enunciado, temos que
		\begin{equation*}
		\Psi(x) (1-\innerproduct{F(x)}{F(y)}) + \innerproduct{\nu(x)}{F(y)} \geq 0.
		\end{equation*}
		
		\item Seja $\{ e_1,e_2 \}$ uma base ortonormal de $T_x \Sigma$ tal que
		\begin{equation*}
		h(e_1,e_1)=\Psi(x), \quad h(e_1,e_2)=0 \quad e \quad h(e_2.e_2)=-\Psi(x)
		\end{equation*}
		
		\item Identificando $F(x)$ com $x$, seja $\gamma$ uma geodésica tal que $\gamma(0)=x$ e $\gamma'(0)=e_1$.
		
		\item Definamos a função $f: \R \rightarrow \R$ tal que
		\begin{equation*}
		f(t) = \Psi(x) (1-\innerproduct{x}{\gamma(t)}) + \innerproduct{\nu(x)}{\gamma(t)} \geq 0.
		\end{equation*}
	\end{itemize}
\end{frame}

\begin{frame}
	\begin{itemize}
		\item Calculando as derivadas de $f$ tem-se:
		\begin{align*}
		f'(t) =& -\innerproduct{\Psi(x)x - \nu(x)}{\gamma'(t)},\\
		f''(t) =& \innerproduct{\Psi(x)x - \nu(x)}{\gamma(t)}\\
		& + h(\gamma'(t),\gamma'(t)) \innerproduct{\Psi(x)x - \nu(x)}{\nu(\gamma(t))},\\
		f'''(t) =& \innerproduct{\Psi(x)x - \nu(x)}{\gamma'(t)}\\
		& + h(\gamma'(t),\gamma'(t)) \innerproduct{\Psi(x)x - \nu(x)}{D_{\gamma'(t)} \nu(\gamma(t))}\\
		& + (D_{\gamma'(t)} h) (\gamma'(t),\gamma'(t)) \innerproduct{\Psi(x)x - \nu(x)}{\nu(\gamma'(t))}.
		\end{align*}
	\end{itemize}
\end{frame}

\begin{frame}
	\begin{itemize}
		\item Avaliando em $0$ temos: $f(0)=f'(0)=f''(0)=0$.
		
		\item Como $f(t) \geq 0$, então $f'''(0)=0$.
		
		\item Portanto, $(D_{e_1}h)(e_1.e_1)=0$.
		
		\item Mudando a orientação à base $\{ e_2, e_1, -\nu \}$, e procedendo de maneira similar, temos que $\parentheses{D_{e_2}h} \parentheses{e_2,e_2}=0$.
		
		\item Usando as equações de Codazzi, podemos concluir que $\nabla h \equiv 0$.
		
		\item Portanto, $h$ é constante, e com isso, a curvatura Gaussiana $K$ é constante.
	\end{itemize}
\end{frame}





\begin{frame}
	\begin{itemize}
%		\item A curvatura Gaussiana de uma superfície em $S^3$ está dada por
%		\begin{equation*}
%		K = 1 + \lambda_1 \lambda_2,
%		\end{equation*}
%		onde $\lambda_1$ e $\lambda_2$ são as curvaturas principais da superfície.
		
		\item Pelo Teorema \ref{curv-gauss-de-sup-min-em-S3}, a curvatura Gaussiana é $K \equiv 1$ ou $K \equiv 0$.
		
		\item Se $K=1$, então $F$ tem pontos umbílicos. 
		Isso é uma contradição.
		
		\item Portanto a curvatura Gaussiana $K \equiv 0$ e $F$ é um pedaço do toro de Clifford.
		Como $F$ é compacta, então $F$ é o toro de Clifford.
		\qed
	\end{itemize}
\end{frame}

\begin{frame}
	Definamos $\aleph$ da seguinte forma:
		\begin{equation*}
			\aleph = \sup_{\substack{x,y \in \Sigma \\ x \neq y}} \frac{\norm{\innerproduct{\nu(x)}{F(y)}}}{\Psi(x) (1-\innerproduct{F(x)}{F(y)})}.
		\end{equation*}


	\begin{observacao}
		Pela Proposição \ref{aleph-leq-1}, quando $\aleph \leq 1$ temos que a superfície é o toro de Clifford. 
	\end{observacao}
\end{frame}

\begin{frame}
	\begin{observacao}
		A partir de agora vamos analisar o caso onde $\aleph > 1$.
	\end{observacao}
	
	Da definição de $\aleph$, podemos ter a função $Z: \Sigma \times \Sigma \rightarrow \R$ descrita da forma:
	\begin{equation*}
		Z(x,y) = \aleph \Psi(x) \left(1 - \innerproduct{F(x)}{F(y)} \right) + \innerproduct{\nu(x)}{F(y)}.
	\end{equation*}
	
	Definamos também o conjunto
	\begin{equation*}
	\Omega = \left\{ x \in \Sigma: \exists y \in \Sigma \setminus \{ x \}, Z(x,y)=0 \right\}.
	\end{equation*}
\end{frame}



\begin{frame}
	\begin{proposicao}
		$\Omega$ não é vazio.
	\end{proposicao}
\end{frame}

\begin{frame}
	\begin{proposicao}
		A função $Z$ cumpre a seguinte desigualdade:
		\begin{multline*}
			\sum_{i=1}^{2} \frac{\partial^2 Z}{\partial x_i^2}(x,y) + 2 \sum_{i=1}^{2} \frac{\partial^2 Z}{\partial x_i \partial y_i}(x,y) + \sum_{i=1}^{2} \frac{\partial^2 Z}{\partial y_i^2}(x,y) \leq \\
			- \frac{\aleph^2 -1}{\aleph} \frac{\Psi(x)}{1 - \innerproduct{F(x)}{F(y)}} \sum_{i=1}^{2} \innerproduct{\frac{\partial F}{\partial x_i}(x)}{F(y)}^2 \\ 
			+ \overline{\Lambda}(x,y) \left( Z(x,y) + \sum_{i=1}^{2} \left| \frac{\partial Z}{\partial x_i}(x,y) \right| + \sum_{i=1}^{2} \left| \frac{\partial Z}{\partial y_i}(x,y) \right| \right),
		\end{multline*}
		onde $\overline{\Lambda}(x,y)$ pode não ser limitada em uma vizinhança da diagonal.
	\end{proposicao}
\end{frame}

\begin{frame}
	\begin{teorema}[\cite{Brendle2010}]
		\label{bony's-strict-maximum-principe}
		Seja $\Omega$ um subconjunto aberto de $\R^n$, e sejam $X_1, \ldots, X_m$ campos vetoriais diferenciais em $\Omega$. Assuma que $\varphi: \Omega \rightarrow \R$ é uma função diferenciável não negativa satisfazendo
		\begin{equation*}
			\sum_{j=1}^{m} D^2 \varphi (X_j,X_j) = -L \inf_{|\xi| \leq 1} D^2 \varphi(\xi,\xi) + L \varphi + L |D \varphi|,
		\end{equation*}
		onde $L$ é uma constante positiva. Seja $F= \{ x \in \Omega: \varphi(x)=0 \}$ o conjunto de zeros da função $\varphi$. Adicionalmente, suponha que $\gamma: [0,1] \rightarrow \Omega$ é um caminho diferenciável tal que $\gamma(0) \in F$ e $\gamma'(s) = \sum_{j=1}^{m} f_j(s) X_j(\gamma(s))$ para funções diferenciáveis $f_1, \ldots, f_m: [0,1] \rightarrow \R$. Então $\gamma(s) \in F$ para tudo $s \in [0,1]$.
	\end{teorema}
\end{frame}

\begin{frame}
	\begin{observacao}
		O Teorema \ref{bony's-strict-maximum-principe} é válido ainda quando $\Omega$ é um aberto de uma variedade Riemanniana. 
	\end{observacao}

		\begin{proposicao}
		$\Omega$ é aberto.
	\end{proposicao}
\end{frame}




\begin{frame}
	\frametitle{Referências}
	\bibliography{references}
\end{frame}

\end{document}